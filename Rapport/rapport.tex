\documentclass[11pt,a4paper]{article}
\usepackage[utf8]{inputenc}
\usepackage{amsmath}
\usepackage{amsfonts}
\usepackage{amssymb}
\usepackage{graphicx}
\begin{document}
\title{Rapport projet 2: Architecture}
\maketitle

\section*{Architecture globale}
Notre implémentation se déroulerait en quatre phase:

\begin{itemize}
\item{Calcul des différents nombres premiers sur 32bits utiles à l'algorithme de factorisation}
\item{Lecture des différents fichier accompagnée du création d'une liste contenant les nombres chargés}
\item{Factorisation des nombres chargé en nombres premiers ainsi qu'une mise à jours des comptes des nombres premiers ainsi trouvés}
\item{Recherche du nombre premier ayant un compte valant 1}
\end{itemize}

La première et deuxième phase se déroulerait de façon simultanée: le programme principal lancerait des threads qui chargerait les nombres des fichiers en les stockant dans une liste chainée (située sur le tas). Pendant le chargement des nombres, il calculerait l'ensemble des nombres premiers sur 32bits (cfr Algorithme 1) et les stockerait dans un tableau à deux dimensions (situé sur le tas). La première dimension représentait le nombre premier, la seconde servirait de compte. 

Une fois les deux premières phases terminées, le programme principal relancerait une série de threads qui enlèverait les nombres un à un de la liste chainée pour ensuite les factorier (cfr. Algorithme 2) en mettant les comptes à jours dans le tableau à deux dimensions. Les thread se terminent lorsque la liste chainée est vide. 

La quatrième phase commence lorsque le programme principal a repérée la terminaison des différents threads de la phase trois. Il relance alors des thread qui vont chacun parcourir une partie du tableau de compte à la recherche du nombre premier ayant un compte valant 1. Ce nombre est alors retourné au programme principal. 

\section*{Mécanisme de synchronisation}
Les différents threads actualiseront des données sur le tas (tableau à deux dimension de compte, liste chainée de fichier et lsite chainées de nombres). Ils y accéderont au moyen de l'exclusion mutuelle en utilisant des mutex/sémaphores.

\section*{Structure de données}
\begin{itemize}
\item{Liste chainée: Fichier et nombres chargé}
\item{Tableau de compte}
\end{itemize}

\section*{Algorithmes}
 \subsection*{Algorithme 1: génération de nombres premier}
 
 On suppose que 2 est premier. On va tester chaque nombre l'un à la suite de l'autre en commençant par trois. Le test est le suivant: un nombre est premier s'il n'est pas divisible par un nombre premier inférieur à se racine carrée (nombre donc déjà calculé et enregistré).
 
 \subsection*{Algorithme 2: Factorisation en produit de nombre premier}   
 
 On procède par itération. On test le nombre afin de voir si il est divisible par le plus petit nombre premier trouvé. Si c'est le cas, on considère que ce nombre premier est facteur et on réitère le test sur le résultat de la division ainsi obtenu. Si le test venait à échouer, le test de division s'effectue avec le nombre premier suivant. 
\end{document}