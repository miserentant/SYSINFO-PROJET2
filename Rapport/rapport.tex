\documentclass[11pt,a4paper]{article}
\usepackage[utf8]{inputenc}
\usepackage{amsmath}
\usepackage{amsfonts}
\usepackage{amssymb}
\usepackage{graphicx}
\begin{document}
\author{Iserentant Merlin - Momin Charles}
\title{Rapport projet 2: Architecture}
\maketitle

\section{Architecture globale}
Le programme se déroulera en trois phases:

\begin{itemize}
\item{Chargement des nombre issus des fichiers}
\item{Factorisation des nombres}
\item{Analyse des facteurs premiers}
\end{itemize}

Les deux premières phases se dérouleraient simultanéement dans un schéma de producteur/consommateur. Au lancement du programme, des threads producteurs seraient créés afin de lire les nombres contenus dans les différents fichiers et les stocker dans un tableau. "N" threads seraient lancés de façon simultanée afin de factoriser les nombres ainsi obtenus. La factorisation se ferait se ferait grâce à l'algorithme 2. L'analyse quant à elle serait faite dans un deuxième temps par le programme principal qui n'aurait qu'à vérifier les comptes des nombres premiers ainsi trouvés. 

\section{Mécanisme de synchronisation}
Les producteur/consommateurs actualiseront un tableau de données qui leurs servira de lieu d'échange. Tout ces échanges seront effectués via un mécanisme de mutex et sémaphore.

\section{Structure de données}
\begin{itemize}
\item{Tableaux de nombres (zones d'échanges entre producteur et consommateur)}
\item{Liste chainée: Liste des nombres premiers et leurs comptes. Elle est mise à jour si nécessaire.}
\end{itemize}

\section{Algorithmes}
 \subsection{Algorithme 1: génération de nombres premier}
 
Pour tester si un nombre est premier, il suffit de tester s'il est divisible par tous les nombres premiers inférieur à sa racine carré (déjà calculés et stockés dans la liste chainée). Si ce n'est pas le cas, il s'agit bien d'un nombre premier. 
 
 
 \subsection*{Algorithme 2: Factorisation en produit de nombre premier}   
 
 On procède par itération. On test le nombre afin de voir si il est divisible par le plus petit nombre premier (2). Si c'est le cas, on considère que ce nombre premier est facteur et on réitère le test sur le résultat de la division ainsi obtenu. Si le test venait à échouer, le test de division s'effectue avec le nombre premier suivant (si la liste chainée n'indique pas de nombre premier suivant, celui-ci est calculé à l'aide de l'algorithme 1 et rajouté à la liste chainée). Les tests de division s'arrête lorsque le nombre est devenu égale à l'unité.
\end{document}

